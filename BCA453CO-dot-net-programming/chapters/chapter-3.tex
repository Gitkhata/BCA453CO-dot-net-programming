\chapter{Developing Console Application}

A console application is an application that takes input and displays output at a command line console with access to three basic data streams: standard input, standard output and standard error. 

A console application facilitates the reading and writing of characters from a console — either individually or as an entire line. It is the simplest form of a {\cs} program and is typically invoked from the Windows command prompt. A console application usually exists in the form of a standalone executable file with minimal or no graphical user interface (GUI).

The program structure of a console application facilitates a sequential execution flow between statements. Designed for the keyboard and display screen, a console application is driven by keyboard and system events generated by network connections and objects.

A console application is primarily designed for the following reasons:
\begin{itemize}
	\item To provide a simple user interface for applications requiring little or no user interaction, such as samples for learning {\cs} language features and command-line utility programs.
	\item Automated testing, which can reduce automation implementation resources.
\end{itemize}

Console applications have one main entry point (static main method) of execution, which takes an optional array of parameters as its only argument for command-line parameter representation.

The \verb|.NET| Framework provides library classes to enable rapid console application development with output display capability in different formats. \verb|System.Console| (a sealed class) is one of the main classes used in the development of console applications.

\section{Entry Point Method}

The \verb|Main| method is the entry point of a {\cs} application. (Libraries and services do not require a Main method as an entry point.) When the application is started, the Main method is the first method that is invoked. Main is declared inside a class or struct. Main must be static and it need not be public. 

There can only be one entry point in a {\cs} program. If you have more than one class that has a Main method, you must compile your program with the main compiler option to specify which Main method to use as the entry point.
\begin{lstlisting}[numbers=none]
// entry point method
class EntryPointMethod {
static void Main(string[] args) {
// Display the number of command line arguments.
	Console.WriteLine(args.Length);
	}
}   
\end{lstlisting}

Where,
\begin{itemize}
	\item \verb|static|: It means Main Method can be called without an object.
	\item \verb|public|: It is access modifiers which means the  compiler can execute this from anywhere.
	\item \verb|void|: The Main method doesn’t return anything.
	\item \verb|Main()|: It is the configured name of the Main method.
	\item \verb|string[] args|: For accepting the zero-indexed command line arguments. \texttt{args} is the user-defined name. So you can change it by a valid identifier. \verb|[]| must come before the args otherwise compiler will give errors.
\end{itemize}



\section{Command Line Parameters}
In some situations, you have to send command line parameters to your application. When you pass arguments, the command line input, read from the standard entry point as string array.
You can send arguments to the Main method by defining the method in one of the following ways:

\begin{lstlisting}[numbers=none]
static int Main(string[] args)
\end{lstlisting}

OR

\begin{lstlisting}[numbers=none]
static void Main(string[] args)
\end{lstlisting}

The arguments of the Main method is a String array that represents the command-line arguments. Usually you determine whether arguments exist by testing the \texttt{Length} property.

\begin{lstlisting}[numbers=none]
if (args.Length == 0) {
	System.Console.WriteLine("No arguments found!!");
	return 1;
}
\end{lstlisting}

Listing \ref{lst:command-line-args} shows the use of command line argument. 
\lstinputlisting[caption=Example of command line argument, label={lst:command-line-args}]{CommandLineArgumentExample.cs}

\section{Compiling and Building Projects}
The code below will demonstrate a {\cs} program written in a simple text editor. Start by
saving the following code to a text file called \verb|hello.cs|:

\lstinputlisting[caption=Compiling Csharp program] {CompileCsharpCodeCommandLine.cs}

To compile \verb|hello.cs|, run the following from the command line:

\begin{itemize}
\item For standard Microsoft installations of .\ NET 4.7, run 
\begin{verbatim}
c:\windows\microsoft.net\framework\v4.7.0\csc.exe hello.cs
\end{verbatim}
\item For Mono run \verb|mcs hello.cs|
\end{itemize}

Doing so will produce hello.exe. The following command will run hello.exe:
 
\begin{itemize}
\item On Windows, use \verb|hello.exe| using CMD/Powershell
\item On Linux, use \verb|mono hello.exe| using Terminal
\end{itemize}


\section{Using control Statements in Console Application}
\lstinputlisting[caption=Using control statements in console application] {CommandLineFactorial.cs}

\newpage\thispagestyle{empty}
